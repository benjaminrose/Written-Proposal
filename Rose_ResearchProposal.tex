% 2016 Written PhD Proposal
% Benjamin Rose, advised by Peter Garnavich

%somehow this does not understand \text{}

% \documentclass{aastex}
% \documentclass[preprint2]{aastex}
% default is double spaced `manuscript`, `preprint` or `preprint2` are options
\documentclass[apj, iop]{emulateapj}
\usepackage{graphicx}% Include figure files
\usepackage{color}%for preprints
\usepackage[dvipsnames]{xcolor} %allows for gray links
\usepackage[colorlinks, linkcolor=blue, citecolor=blue, urlcolor=blue]{hyperref}
\usepackage{cleveref} %this allows for \cref{}: http://texblog.org/2013/05/06/cleveref-a-clever-way-to-reference-in-latex/
\usepackage{natbib}
\usepackage{amsmath}  %needed for $\text{}$ and aastex

\newcommand{\sn}{SNIa}
\newcommand{\todo}[1]{\textbf{\textcolor{red}{#1}}}
\newcommand{\lcdm}{$\Lambda$CDM}     %should use ensure math or something.
\newcommand{\kms}{\ensuremath{~\text{km s}^{-1}}}

% running headers for journal/emulate
\shorttitle{Testing the Cosmological Principle} 
\shortauthors{Rose}

% adds date to aastex versions (emulate has this by default)
% \slugcomment{draft version: \today}
% \slugcomment{}

\begin{document}

\title{Precision Cosmology: Testing the Cosmological Principle \\with Improved Supernova Distances}

\author{Benjamin Rose, {\it advised by} P. M. Garnavich}
\affil{University of Notre Dame, Center for Astrophysics, Notre Dame, IN 46556}
\email{brose3@nd.edu}

% \and
% \author{{\it advised by} P. M. Garnavich}
% \affil{University of Notre Dame, Center for Astrophysics, Notre Dame, IN 46556}

\begin{abstract}
I am looking for at improving cosmology with \sn{}
\end{abstract}

\maketitle

\section{Introduction}\label{introduction} 

Type Ia supernova (\sn{}) have been a standard cosmological tool since they were
successfully used to estimate the Hubble constant \citep{Hamuy95,Riess95}, and
later see the accelerated expansion of the Universe
\citep{Riess98,Perlmutter99}. Cosmology is now focusing on precision and with
improved precision we are able to test a fundamental assumption of cosmology:
that the universe is homogeneous and isotropic on the largest scales. In order
to check the Cosmological Principle, both the statistical and systematic
uncertainties in \sn{} distances must be reduced. These measurements will give
deeper understanding of the evolution of the cosmos and the structure and
composition of the universe.

Simple tests can be constructed to see if there is a consistent cosmology in all
directions. For these to reach out to distances of gigaparsecs, or of order
$0.1$ in redshift, \sn{} need to have very accurate distance measurements. \sn{}
distances are already corrected for light curve width and color variations, but
further new corrections appear to be important distance accuracy, such as
correlations with host galaxy properties or even the local environment near the
explosion. Procuring these tests will \todo{finish my thesis statement.}

\section{Testing the Cosmological Principle}\label{testing-the-cosmological-principle}

The excepted cosmological model, $\Lambda$CDM, works under a testable assumption
that the universe is isotropic and homogeneous on large scales, scales greater
than the size of the present day first acoustic peaks seen in the CMB.
%(\cite{Scrimgeour12} says greater-equal 100h$^{-1}$Mpc in ΛCDM).   
We already
know that gravity breaks this isotropic condition on small scales and produces
peculiar motion of galaxies. Specific theories of inflation (like in
\cite{MersiniHoughton:2008io}) produce a preferred direction in our visible
universe due to a slight curvature in space-time from pre-inflation
interactions. \todo{go into theory detail?} Observationally, if this preferred
direction exists, we expect to see a  uniform bulk velocity flow that extends
outside the local group and into what $\Lambda$CDM would predict to be a purely
isotropic Hubble expansion. Such a large-scale asymmetry has been called a `dark
flow'. \todo{by whom, Kashlinsky's first paper or his ``review'' paper?}

\subsection{Previous Tests}\label{previous-tests}

The universe's isotropic nature has been tested, over the last decade, with
surveys of nearby galaxies \citep{Ma13}, kinetic Sunyaev-Zel\'{d}ovich (kSZ)
effect on distant clusters \citep{Kashlinsky10,Planckdf}, and with \sn{} data
sets \citep[and others as seen in \Cref{t:review}]{Dai11,Rathaus13}. Most
searches did not see a bulk flow that was significantly larger then allowed by
\lcdm{} but some saw a bulk flow of greater then $1000 \kms{}$ out to
redshifts of $z = 0.25$. This measurement cannot be explained by $\Lambda$CDM
alone but needs an inflation theory that predicts this cosmic anisotropy.  A
summary of the results from past searches can be found in \Cref{t:review}.
\todo{remove both table references here.}

\todo{kaslinskly plot of measurements and what LCDM can allow.}

\todo{other plot from aps presentation}

\begin{figure}
	\includegraphics[width=3.2in]{what_dataset_size_v_velocity.pdf} 
    \caption{The number of \sn{} needed to see a given dark flow velocity, a
	cosmic scale bulk flow, out to a given redshift. It is seen that Union2.1 
	can detect flows only out to $z=0.05$ and nearly 10$^4$ \sn{} would be 
	needed to see a dark flow out to $z=0.3$. Taken from \cite{Mathews16}}
	\label{f:sn-needed} 
\end{figure}

My work on initial tests of the Cosmological Principle can be found in
\cite{Mathews16}. I worked heavily on developed a new method to search for
cosmological asymmetries using the angular separation of a \sn{} and the
direction of the dark flow. We were able to minimize the data fitting both
\lcdm{} cosmological variables and  an bulk flow. If the same bulk flow is
visible across all redshifts, then we have found the dark flow signal. This
analysis was performed on the Union2.1 \citep{Suzuki12} \sn{} sample. We found a
mild bulk flow of $325 \pm 54$ km s$^{-1}$ out to $z = 0.05$. This is consistent
with most past measurements and does not contradict \lcdm{}.

From further analysis, we determined that the loss of signal past $z \sim 0.05$
was not from the lack of a dark flow, but from the increase of the uncertainty
in the distance measurements. An order of magnitude reduction of distance
modulus error is needed to see out to $z \sim 0.3$. This is unattainable with
the current data sets, but we propose two ways to attack this problem. First,
further data sets will reduce the statistical error in \sn{} distances over
large parts of the sky. This relationship is illustrated in \Cref{f:sn-needed}.
This large sample size should be achievable with the Large Synoptic Survey
Telescope (LSST) and its uniform photometric reliability of $10 ~\text{mmag}$
across the visible sky \citep{Ivezic08}. A large, uniform data set will push the
statistical errors at $z\sim 0.3$ to the systematic floor. Second, we propose to
lower the systematic floor by including the properties of the local host
environment in correcting \sn{} distances.

\subsection{Future Tests}\label{future-tests}

Testing the Cosmological Principle can and should be continued. In search for a
cosmic anisotropy, studies have used galactic surveys, the kinetic Sunyaev-
Zel\'dovich effect, and SNIa, but another possibility is using baryon acoustic
oscillations (BAO). This distance measure is complementary to \sn{} estimates
and subject to smaller systematic uncertainties.
The BAO analysis will follow a similar procedure as developed in
\cite{Mathews16} for SNIa. The test requires a
data set with significant sky coverage, and this is possible with SDSS's eBOSS
and eventually DESI.

% Another alternative is future surveys like WFIRST and LSST. There will still be 
% a long wait for data from either of these surveys. 
% Later it can be done with WFIRST (with its some large sky fraction.)
% and/or LSST data.

\cite{Mathews16} shows that LSST will produce a data set that is able to search for
the cosmic dark flow. Detailed simulations of LSST-like data sets are critical in understanding
the limits of the survey when applied to the dark flow problem. These tests will show if the combination of
LSST's sensitivity, cadence, and sky coverage will be sufficient to test
for isotropic expansion as predicted by $\Lambda$CDM and what limits can be
placed on the dark flow.
WFIRST may also be able to help in this measurement, however, its sky
coverage will be limited compared with LSST.
Simulations of the combined LSST and WFIRST data sets will produce useful insights into
optimum future surveys.

These large surveys are being leveraged to reduce the statistical uncertainties in distance
measurements. But we also plan to reduce the systematic error floor by studying the
local environments of SNIa.

\todo{talk about difficulties and solutions}

\section{Standardizing type Ia
supernova}\label{standardizing-type-ia-supernova}

The \sn{} community as a whole is focusing on reducing uncertainty in SNIa
distances. Knowing distance more precisely will narrow the range for the Hubble
constant and the dark energy equation of state as well as
constraining the Cosmological Principle.

\sn{} are not standard candles, but rather standardizeable candles. Since
\cite{Phillips93} there has been significant work to improve off this basic correction to the
variability in \sn{} absolute luminosity. \sn{} color is also used to standardize \sn{} and
this has allowed them to be used for critical cosmological measurements
(\citep{Riess98, Perlmutter99}). Currently well-studied \sn{} typically have a
15\%\ error in distance modulus (or 7\% in distance). So it takes about 50 SNIa
to reduce the statistical error down to a few percent in brightness, which is near the 
the systematic floor. This floor is partly caused by limits on photometric calibration and
partly by intrinsic uncertainties in the supernovae themselves.

\subsection{New Standardizations}\label{new-standardizations}

A significant amount of research time and energy is spent on trying to reduce
the scatter in the absolute magnitude of SNIa. It has been found that a
reduction in this can be seen if the properties of the SNIa's host galaxy are
accounted for. Host mass \citep{Childress13} and host metallicity \citep{Hayden13}
have been seen to reduce this intrinsic scatter.

The idea that these global properties directly effect the \sn{} is a bit of a
stretch, so the community has started to look at the local environment around
the SNIa. \cite{Rigault13} looked at \sn{} form the Nearby Supernova Factory and
looked at H$_{\alpha}$ within a 1 kpc radius of each SNIa. They found  a bias in
standardized brightness of the SNIa. Also \cite{Rigault15, Jones15} looked at
\sn{} in the Constitution sample with host galaxy data from {\it GALEX} but they
disagree on whether there is or is not a bias. Sorting this controversy out is
critical in improving the estimate of the Hubble constant.
%What is visible is at best at a $2\sigma$ level. %Jones15 quotes 2-4\sigma significance

\subsection{Using HST}\label{using-hst}

\todo{combine HST and future work sections}

I am currently working with Hubble Space Telescope (HST) images of host galaxies
of \sn{} found during SDSS-II Supernova Survey. HST's small angular resolution
allows for a much smaller definition of local environment. For a galaxy at $z =
0.1$ HST can get an environment of $\sim$160 pc, verses the 1 kpc that was used 
with {\it GALEX} data in \cite{Jones15,Rigault15}.

This set of HST images will give us a few variables for the local environment
that we might be able to use for Corrections to the SNIa's absolute brightness.
The first being the SNIa's fractional pixel rank (FPR). The FPR is the ranked
fraction, by brightness, of the pixel at the SNIa's position relative to the
rest of the galaxy. This gives us a proxy for star formation. Another variable
would be the color of the SNIa's pixel. These two variables will be compared to
the residuals on the Hubble diagram to look for a bias. Further analysis of sub-
sets will be done. We will be able to answer questions like whether there is a
bias for \sn{} in clumpy regions of elliptical galaxies.

\begin{figure}
	\includegraphics[width=3.2in]{SN2635-combined-inverted.pdf}
	\caption{Host galaxy of SDSS SN2635 (2005fw), at a redshift of z=0.143. HST 
	(left) one pixel is 114 pc, and SDSS (right) one pixel is 1,140 pc.}
	\label{f:galaxy-compare}
\end{figure}

Understanding the local environment can be done with \sn{} in the local universe,
but that has some issues. First the number of \sn{} is proportional to the volume
for which you are looking, and there is more volume at higher redshift. Secondly
there are questions about whether \sn{} are the same across cosmic time, so this
test should be done for as wide a range of redshifts as possible. For these
reasons, SDSS's $z \sim 0.1$ is a good range to perform these tests. At this
redshift, HST's high angular resolution is the best for observing a true local
environment.

% F derivative R - (second derivative) - finds things that do not match 
% its neighbors (nor follows a nice smooth slope like ellipticals)

% a color - Yea color-mag diagram \& ``type/max age'' of stellar region

\subsection{Future Work}\label{future-work-1}

We plan several ways to study the local environment of SNIa. SDSS-IV's MaNGA is using integrated
field units (IFUs) to get spectra at spacial scales of $\sim 2$ kpc. IFUs are a
common technique and their resolution will only get smaller with time.
Alternative IFU's can be used, including Keck OSIRIS that observes in the
infrared with a spacial element similar to HST (0.020"-0.100") \citep{OSIRIS},
% \url{http://ifs.wikidot.com/instruments#toc12}) 
or Gemini GMOS that observes in
the optical but has a much larger resolution 0.2" \citep{Gemini}.
% \url{http://ifs.wikidot.com/instruments#toc7})
Proposals have been submitted to 
get IFU spectra on sub-sets of the host galaxies using MaGNA..

% Look spectra!
% This will not be the end of local environment tests. SDSS MaNGA is using 
% integrated field units (IFUs) to get spectra at scales of $\sim 2$ kpc. IFUs
% are a common technique and their resolution will only get smaller with time.

More photometry will also be useful. There are more hosts that HST can observe.
Around 1000 \sn{} were seen with SDSS \citep{Campbell13} but there are only
around 60 in HST's archive. There is a large number of targets that still could be
observed to increase the data set and improve our statistical uncertainties.
Infrared photometry could also take advantage of ground based adaptive optic
telescopes.

\todo{talk about difficulties and solutions}

\todo{conclusion}

% A slit on the SN location. (idk if this is done, or useful?)

% Ground based observations of hosts

% More HST: filters of these, more statistics, . . .

\bibliographystyle{apj}
\bibliography{2016-02-07-w-ResearchProposal}

\begin{deluxetable}{ccc|cc|c|cc}
\tablecolumns{8}
% \tabletypesize{\small} %equal to title size, seems too big
% \tabletypesize{\footnotesize} %can split after Rathaus13
\tabletypesize{\scriptsize} %one line (+ footnotes) too long for one page
% \tabletypesize{\tiny} %fits on one page
\tablewidth{0pt} %its natural width, (fix different natural widths with info from log file)
% \tablewidth{556pt} %from log tables are 524.95901pt & 555.51466pt wide. But this look strange.
% \rotate
\tablecaption{Summary of bulk flow searches.\label{t:review}} %really a title
\tablehead{
	\colhead{Reference} & \colhead{Obj. Type} & \colhead{No. Obj.} & \colhead{Redshift\tablenotemark{a}} & \colhead{Distance\tablenotemark{a}}  & \colhead{$v_{bf}$} & \colhead{$l$} & \colhead{$b$} \\
	\colhead{}			& \colhead{}		  & \colhead{}		   & \colhead{}		 					 & \colhead{h$^{-1}$ Mpc}				& \colhead{km s$^{-1}$} 	& \colhead{deg}	& \colhead{deg}
	}

\startdata
\cite{Kashlinsky10}  & kSZ & 516 & $< 0.12$ & $< 345$ & $934  \pm 352$ & $282 \pm 34$ &  $22 \pm 20$ \\
                            & & 547 & $< 0.16$ & $< 430$ & $1230 \pm 331$ & $292 \pm 21$ & $27 \pm 15$ \\
                            & & 694 & $< 0.20$ & $< 540$ & $1042 \pm 295$ & $284 \pm 24$ & $30 \pm 16$ \\
                            & & 838 & $< 0.25$ & $< 640$ & $1005 \pm 267$ & $296 \pm 29$ &  $39 \pm 15$ \\ 
\tableline
\cite{Dai11} & SN Ia & 132 & $< 0.05$ & $< 145$ &  $188 \pm 120$  &  $290 \pm 39$ & $20 \pm 32$ \\
                                      & & 425 & $> 0.05$ & $> 145$ &  \nodata     &   \nodata    &   \nodata   \\
\tableline
\cite{Weyant11} & SN Ia & 112 & $< 0.028$ & $< 85 $ & $538 \pm  86$ & $250 \pm 100$ & $36 \pm 11$ \\ 
\tableline
\cite{Ma11}   & \begin{tabular}{c}galaxies \\ \& SN Ia \end{tabular} & 4536 & $< 0.011 $ & $< 33$ &    $340 \pm 130$ & $285 \pm 23 $ &$ 9 \pm 19$ \\ 
\tableline
\cite{Colin11} & SN Ia & 142 & $< 0.06$ & $< 175$ & $260 \pm 130$ & $298 \pm 40 $ &$ 8 \pm 40$ \\
\tableline
\cite{Turnbull12} & SN Ia & 245 & $< 0.05$ & $< 145$ & $245 \pm  76$ & $319 \pm 18 $ &$ 7 \pm 14$ \\ 
\tableline
\cite{Feindt13} & SN Ia  & 128 & 0.015 - 0.035 & 45 -108 & $243 \pm  88$ & $298 \pm 25$ &$15 \pm 20$ \\ 
                                             & & 36  & 0.035 - 0.045 & 108 - 140 & $452 \pm 314$ & $302 \pm 48$ & $ -12 \pm 26$ \\
                                             & & 38  & 0.045 - 0.060 & 140 - 188 & $650 \pm 398$ & $359 \pm 32$ & $ 14 \pm 27$ \\
                                             & & 77  & 0.060 - 0.100 & 188 - 322 & $105 \pm 401$ & $285 \pm 234$ & $ -23 \pm 112$ \\ 
\tableline
% dist is correct, z is a guess
\cite{Ma13}       & galaxies & 2404 & $< 0.026$ & $< 80$ & $280 \pm 8$ & $280 \pm 8 $ &$ 5.1 \pm 6$ \\ 
\tableline
\cite{Rathaus13} & SN Ia & 200 & $< 0.2$ &  $< 550$ & $260$ & $295$ &$ 5$ \\ 
\tableline

% \tablebreak %splits the table over two pages here

%add \citet{Wiltshire13}? & galaxies &
\cite{Appleby14} & SN Ia & 187 & 0.015 - 0.045  & 45 - 130 & \nodata & $276 \pm 29$ &$ 20 \pm 12$ \\ 
\tableline 
\cite{Planckdf} & kSZ &  95 & 0.01 - 0.03 & 30 - 90 & $< 700$ & \nodata & \nodata \\ 
                     & & 1743& $< 0.5$  & $< 2000$ & $< 254$ & \nodata & \nodata\\ 
%Planck number from http://arxiv.org/pdf/1007.1916v1.pdf and from its datafile
\tableline
\cite{Mathews16} & SN Ia & 191 & $< 0.05$ & $<  145$ & $325\pm 54$ & $276 \pm 15$ &$ 37 \pm 13$ \\ 
                                &   & 387 & $> 0.05$ & $> 145$ & $ 460 \pm 260$ & $180 \pm 34$ &$ 65 \pm 340$ \\ 
\enddata

\tablenotetext{a}{Distances and redshifts are either the maximum, or a characteristic value if available from the original source. If distance and redshift were not both given in the literature, calculated distances vs. redshift were done with WMAP parameters: $\Omega_M = 0.288$ and $\Omega_{\Lambda} = 0.712$.}
\end{deluxetable}


\end{document}